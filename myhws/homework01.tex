% Search for all the places that say "PUT SOMETHING HERE".

\documentclass[11pt]{article}
\usepackage{amsmath,textcomp,amssymb,geometry,graphicx,enumerate}

\def\Name{ lijingqi }  % Your name
\def\SID{ None }  % Your student ID number
\def\Homework{ HW01 } % Number of Homework
\def\Session{Spring 2020}


\title{CS170--Spring 2020 --- Homework \Homework Solutions}
\author{\Name, SID \SID}
\markboth{CS170--\Session\  Homework \Homework\ \Name}{CS170--\Session\ Homework \Homework\ \Name}
\pagestyle{myheadings}
\date{}

\newenvironment{qparts}{\begin{enumerate}[{(}a{)}]}{\end{enumerate}}
\def\endproofmark{$\Box$}
\newenvironment{proof}{\par{\bf Proof}:}{\endproofmark\smallskip}

\textheight=9in
\textwidth=6.5in
\topmargin=-.75in
\oddsidemargin=0.25in
\evensidemargin=0.25in


\begin{document}
\maketitle

Collaborators: NONE

\section*{1. Q1}
pass
\section*{2. Q2}
pass
\section*{3. Q3}
pass
\section*{4. Asymptotic Complexity Comparisons}

\begin{qparts}
\item answer:

note there are some equation

$ f_4(n) =2^{\log_{2}{n}} = n $

$ f_5(n) = \sqrt{n} =  n^{\frac{1}{2}} $

so result is :

$ f_3,f_7,f_2,f_5,f_4,f_9,f_8,f_6,f_1 $

\item answer:

\begin{enumerate}[i.]
    \item
    \begin{equation}
        f(n) = \log_3{n} = \frac{\lg{n}}{lg{3}}  \label{1}
    \end{equation}

    \begin{equation}
        g(n) = \log_4{n} = \frac{\lg{n}}{lg{4}}  \label{2}
    \end{equation}
    
    because equaiton (1) and (2) diff only a constant factor so  $ f(n) = \Theta(g) $
    
    \item
    \begin{equation}
        f(n) = n\log{n^4} = 4n\log{n}  \label{3}
    \end{equation}
    
    \begin{equation}
        g(n) = n^2\log{n^3} = 3n^2\log{n}  \label{4}
    \end{equation}
    
    because equation (4) has the higher degree, so $ f(n) = \mathcal{O}(g) $
    
    \item
    \begin{equation}
        f(n) = \sqrt{n} = n^{\frac{1}{2}} \label{5}
    \end{equation}
    
    \begin{equation}
        g(n) = (\log{n})^3  \label{6}
    \end{equation}
    
    because any polynomial dominates a product of logs, so $ f(n) = \Omega{(g)} $
    
    \item
    \begin{equation}
        f(n) = n + \log{n} \label{5}
    \end{equation}
    
    \begin{equation}
        g(n) = n + (\log{n})^2  \label{6}
    \end{equation}
    
    both f and g grow as $ \Theta(n) $ because the linear term dominates the other, so  $ f(n) = \Theta(g) $

\end{enumerate}
\end{qparts}

\section*{5. Computing Factorials}
pass

\section*{6. Polynomial Evaluation}
pass

\end{document}
