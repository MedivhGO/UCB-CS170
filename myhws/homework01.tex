% Search for all the places that say "PUT SOMETHING HERE".

\documentclass[11pt]{article}
\usepackage{amsmath,textcomp,amssymb,geometry,graphicx,enumerate}
\usepackage{color}
\usepackage{listings}
\lstset{ %
language=C++,                % choose the language of the code
basicstyle=\footnotesize,       % the size of the fonts that are used for the code
numbers=left,                   % where to put the line-numbers
numberstyle=\footnotesize,      % the size of the fonts that are used for the line-numbers
stepnumber=1,                   % the step between two line-numbers. If it is 1 each line will be numbered
numbersep=5pt,                  % how far the line-numbers are from the code
backgroundcolor=\color{white},  % choose the background color. You must add \usepackage{color}
showspaces=false,               % show spaces adding particular underscores
showstringspaces=false,         % underline spaces within strings
showtabs=false,                 % show tabs within strings adding particular underscores
frame=single,           % adds a frame around the code
tabsize=2,          % sets default tabsize to 2 spaces
captionpos=b,           % sets the caption-position to bottom
breaklines=true,        % sets automatic line breaking
breakatwhitespace=false,    % sets if automatic breaks should only happen at whitespace
escapeinside={\%*}{*)}          % if you want to add a comment within your code
}

\def\Name{ lijingqi }  % Your name
\def\SID{ None }  % Your student ID number
\def\Homework{ HW01 } % Number of Homework
\def\Session{Spring 2020}


\title{CS170--Spring 2020 --- Homework \Homework Solutions}
\author{\Name, SID \SID}
\markboth{CS170--\Session\  Homework \Homework\ \Name}{CS170--\Session\ Homework \Homework\ \Name}
\pagestyle{myheadings}
\date{}

\newenvironment{qparts}{\begin{enumerate}[{(}a{)}]}{\end{enumerate}}
\def\endproofmark{$\Box$}
\newenvironment{proof}{\par{\bf Proof}:}{\endproofmark\smallskip}

\textheight=9in
\textwidth=6.5in
\topmargin=-.75in
\oddsidemargin=0.25in
\evensidemargin=0.25in


\begin{document}
\maketitle

Collaborators: NONE

\section*{1.}
pass
\section*{2.}
pass
\section*{3.}
pass
\section*{4. Asymptotic Complexity Comparisons}

\begin{qparts}
\item answer:

note there are some equation

$ f_4(n) =2^{\log_{2}{n}} = n $

$ f_5(n) = \sqrt{n} =  n^{\frac{1}{2}} $

so result is :

$ f_3,f_7,f_2,f_5,f_4,f_9,f_8,f_6,f_1 $

\item answer:

\begin{enumerate}[i.]
    \item
    \begin{equation}
        f(n) = \log_3{n} = \frac{\lg{n}}{lg{3}}  \label{1}
    \end{equation}

    \begin{equation}
        g(n) = \log_4{n} = \frac{\lg{n}}{lg{4}}  \label{2}
    \end{equation}
    
    because equaiton (1) and (2) diff only a constant factor so  $ f(n) = \Theta(g) $
    
    \item
    \begin{equation}
        f(n) = n\log{n^4} = 4n\log{n}  \label{3}
    \end{equation}
    
    \begin{equation}
        g(n) = n^2\log{n^3} = 3n^2\log{n}  \label{4}
    \end{equation}
    
    because equation (4) has the higher degree, so $ f(n) = \mathcal{O}(g) $

    \item
    \begin{equation}
        f(n) = \sqrt{n} = n^{\frac{1}{2}} \label{5}
    \end{equation}
    
    \begin{equation}
        g(n) = (\log{n})^3  \label{6}
    \end{equation}
    
    because any polynomial dominates a product of logs, so $ f(n) = \Omega{(g)} $
    
    \item
    \begin{equation}
        f(n) = n + \log{n} \label{5}
    \end{equation}
    
    \begin{equation}
        g(n) = n + (\log{n})^2  \label{6}
    \end{equation}
    
    both f and g grow as $ \Theta(n) $ because the linear term dominates the other, so  $ f(n) = \Theta(g) $

\end{enumerate}
\end{qparts}

\section*{5. Computing Factorials}
\begin{qparts}

\item Find an $ f(N) $ so that $ N! $ is $ \Theta(f(N)) $ bits long

m bit number multiply n bit number will get a ( m + n ) bit number

upper bound

$ 1 * 2 * 3 * 4  ...  N $

$ totalbit = \sum_{i=1}^{N}\lg{i} $

$ totalbit < \sum_{i=1}^{N}\lg{N} = N\lg{N} $

lower bound

$ N/2 * (N/2+1) * (N/2+2) * ... N $

because N/2 has least $ \lg{N} -1 $ bits 

$ \lg{N/2} + \lg{(N/2+1)} + \lg{(N/2+2)} .. + \lg{N}  >  (N/2)(\lg{N}-1) $

so the bit number is $ f(n) = N\lg{N} $, this means that our number has  $ \Theta{(N\lg{N})} $ bits.

\item Give a naive algorithm to compute $ N! $

\begin{lstlisting}
long factorial(int N) {
    long res = 1;
    for(int i = 1; i <= N; i++) {
        res *= i;
    }
    return res;
}
\end{lstlisting}

Running time:

because res at most $ N\lg{N} $ bit, i at most $ \lg{N} $ bit, so one iteration unit time is $ N(\lg{N})^2 $.
there are $ n $ iterations, so running time is $ \mathcal{O}(N^2(\lg{N})^2) $.
\end{qparts}

\section*{6. Polynomial Evaluation}

\begin{qparts}

\item Describe a naive algorithm that given $ [a_0,...,a_n] $ and $ x $, computes $ p_(x) $

\begin{lstlisting}
res eval([a0... an], x) {
    res = 0;
    for(int i = 0; i < N; i++) {
        res += ai*x^i;
    }
    return res;
}
\end{lstlisting}

It takes i time to compute $ x^i $ on the $ i $th iteration. There are $ n $ iterations. so $ \sum_{i=0}^{N}i = \mathcal{O}(n^2) $.

\item Horner's method

\begin{lstlisting}
res eval([a0... an], x) {
    res = an;
    for(int i = 1; i <= N; i++) {
        res = res * x + an-i
    }
    return res;
}
\end{lstlisting}

Proof:

at first iteration

$ res_{1} = a_{n-1} + a_nx $

at $k-1$ th iteration

$ res_{k-1} = a_{n-k+1} + x(a_{a-k+2} + ... x(a_{n-1} + xa_n)) $

at $k$ th iteration

$ res_{k} = a_{n-k} + res_{k-1}x $

after $ n $th loop iteration , it will have computed the expression $ a_{0} + x(a_{1} + x(a_{2} + x(a_{3} +...+ x(a_{n-1} + xa_{n})))) = \sum_{i=0}^{N}a_{i}x^{i} $

when compute $ res_{k} $, the $ res_{k-1} $ already has been compute, so every iteration does an addition and a multiplication, spending constant cost, so running time is $ \mathcal{O}(n) $.

\end{qparts}

about (a) and (b) are all Polynomial Evaluation algorithm, we can see algorithm is faster than (a).

\end{document}
